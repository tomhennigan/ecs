\documentclass[10pt,a4paper]{article}

\usepackage[cm]{fullpage}

\usepackage{url}
\usepackage[bookmarksnumbered,unicode,pdfborder=1,breaklinks,colorlinks,linkcolor=blue,urlcolor=blue,citecolor=red]{hyperref}
\newcommand{\mailto}[1]{\href{mailto:#1}{\nolinkurl{#1}}}

\begin{document}

\title{COMP6051: Reflective Summary}
\author{Tom Hennigan (\mailto{th10g08@ecs.soton.ac.uk})}
\maketitle

% =============================================================================
% Author note
% -----------
% Although I've added the titles I don't think they're looking for these in the
% final document as it's meant to be an essay. Personally I'm going to remove
% them, however obviously the choice is yours if you want to or not.
% =============================================================================

% -----------------------------------------------------------------------------
% From https://secure.ecs.soton.ac.uk/noteswiki/w/COMP6051/Portfolio
% -----------------------------------------------------------------------------
% Towards the end of the semester you will be asked to write an individual
% reflective summary of your work on the portfolio. In your summary you should
% reflect on the approaches taken in the project and on your part in executing
% them. This is an individual piece of work that will demonstrate your personal
% understanding of the issues identified by the team.
% 
% Your reflective summary should be a relatively short essay (no more than 1000
% words) that explains the key challenges in your project and how your group
% overcame them (with special attention paid to your personal contribution).
% You are free to structure your essay as you would like. It will be assessed
% based on the following mark scheme:
% -----------------------------------------------------------------------------
% 10% - Presentation and clarity
% 	*	Some marks are reserved for the quality of the essay and your
%		arguments.
%		A strong reflective summary will be very well written and structured,
%		with persuasive points and arguments backed up with evidence.
% 
% 	1.	Critical considerations of societal requirements such as identity,
%		privacy and ownership
% 	2.	Communicate current societal challenges and anticipate emerging
%		challenges
% 	3.	Identify and interpret domain and societal requirements for the
%		deployment of social network solutions
% 	4.	Work in a software development team
% 	5.	Present a technological solution within a broader context
% -----------------------------------------------------------------------------
% Project overview (25%)
% 	*	A strong reflective summary should provide a good summary of the
%		project, including the work-plan and activities, and describe what were
%		the key challenges faced by the project team and how they were overcome
% 
% Learning outcome:
% 	5.	Present a technological solution within a broader context
% -----------------------------------------------------------------------------
% Personal Contribution (40%)
% 	*	A strong reflective summary should provide a clear account of the 
%		author's personal contributions to the project and contain perceptive,
%		objective and self-critical reflections on their engagement with the 
%		project tasks.
% 
% Learning outcome:
% 	4.	Work in a software development team
% -----------------------------------------------------------------------------
% Outcomes (25%)
% 	*	A strong reflective summary should clearly articulate the main outcomes
%		of the project and its key achievements in terms of software, knowledge
%		and people.
% 
% Learning outcomes:
% 	1.	Critical considerations of societal requirements such as identity,
%		privacy and ownership
% 	2.	Communicate current societal challenges and anticipate emerging
%		challenges
% 	3.	Identify and interpret domain and societal requirements for the
%		deployment of social network solutions
% -----------------------------------------------------------------------------

\end{document}
